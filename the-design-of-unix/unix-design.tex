\documentclass[12pt,a4paper]{article}
\usepackage{amsmath}
\usepackage{verbatim}
\usepackage{tabularx}
\usepackage[utf8]{inputenc}
\usepackage[pdftex]{graphicx}
%\usepackage[xetex]{graphicx}
%\usepackage{fontspec,xunicode}
\usepackage{color}

\newcommand{\reversevideo}[1]{\colorbox{black}{\!\phantom{X}\!\!\!\color{white}{#1\!}}}
\fontencoding{T1}
%\pagestyle{empty}
%\renewcommand{\familydefault}{\sfdefault}
%\renewcommand{\familydefault}{\rmdefault}


\begin{document}

\title{\Huge{\textbf{Unix' design}}}
\date{}
\author{}
\maketitle


\section*{Vad är Unix?}

Unix är ett operativsystem från mellan 1969 och 1972,
beroende på hur man räknar. Unix utvecklas fortfarande.
\\

\noindent
Designen bakom Unix har influerat nästan alla
exististerande operativsystem.
\begin{itemize}
\item \textbf{GNU/Linux}\\
Designade för att vara kompatibla med Unix.
Linux är operativsystemskärnan, GNU är alla viktiga program.
GNU kan kombinaras med kärnor från andra Unix-liknande operativsystem,
och Linux med annat än GNU, till exemple Andriod.
\item \textbf{4.3 BSD, FreeBSD, OpenBSD, NetBSD, \ldots}\\
Vidareutvecklingar av Unix och av varandra.
\item \textbf{OS X}\\
Vidareutveckling av NeXTSTEP, FreeBSD och NetBSD.
\end{itemize}
Några exempel vi inte pratade om:
\begin{itemize}
\item \textbf{Minix}\\
Designat efter Unix för att undervisa om Unix. Designat
för att vara lättare att studera än Unix. Nyare version
av Minix är designat för att vara extremt stabilt.
\item \textbf{Xenix OS}\\
Microsofts och SCO:s vidarutveckling version av Unix.
Delar av Xenix har återanvänds för att göra Windows
mer kompatibellt med Unix.
\item \textbf{SunOS, Solaris}\\
Vidigt populära vidarutvecklingar av Unix fron Sun Microsystems (nu Oracle.)
\item \textbf{Haiku}\\
Designat efter BeOS, men är väldigt likt Unix och väldigt kompatibellt med Unix.
\item \textbf{OS/2, eComStation}\\
Inte utvecklat efter Unix, men är delvis kompatibellt med Unix.
\item \textbf{Plan 9 from Bell Labs}\\
Nytt operativsystem utvecklat för att adressera brister i Unix.
Detta har gjorts genom att ta Unix' filosofi till nästa steg.
Det är mer Unix-aktigt än vad Unix är.
\end{itemize}

Det enda idag populära exempelt på ett operativsystem som inte
är Unix-liknande är Windows. Microsoft försöker dock göra
Windows mer likt Unix-liknande operativsystem för att underlätta
för administratörer.
\\

\noindent
(Programspråket C utvecklades för Unix så att det skulle
vara enkelt att få Unix att köras på alla CPU-arkitekturer
istället för endast en, som traditionellt var fallet för
andra operativsystem. Idag programmeras nästan alla
operativsystem i C.)


\section*{Filosofi}

\emph{Simplistisk}. Lösningar ska vara så enkla som möjligt att utveckla.
De behöver dock inte vara enkla att använda, det är dock ofta fallet.
\\

\noindent
\emph{Minimalistisk}. Endast det som verkligen behövs ska vara med.


\section*{Principer}

\emph{stdin--stdout-principen}. Program kommunicerar med varandra
genom att ett program skriver ut och ett annat program läser in
vad det andra programmet skriver ut. Istället för att program
kommunicerar med varandra kan de även kommunicera med användaren
eller läsa från eller skriva till filer. För programmet är det
ingen skillnad mellan att kommunicera med användaren, filer eller
andra program.
\\

\noindent
\emph{Text är universiellt}. Program kommunicerar med varandra
och med användaren på samma sätt. Programmen använder mänskligt
läsbar text. Detta kör det möjligt för användaren att förstå
programmet och programmet behöver inte vet om den kommunicerar
med användaren eller med ett annat program. Det blir också enklare
att studera programmet.
\\

\noindent
\emph{Komposition}. Istället för att skriva ett stort program
skriver man många små program som arbeter tillsammans.
\\

\noindent
\emph{Mångfald}. Program ska gå att använda på andra sätt en
vad som förväntas av programmets utvecklare. Detta realiseras
genom att programmet utför en enda funktion och ingenting annat.
Dena funktion sak vara så liten och väldefineras som möjligt.


\section*{Frågeställning}

Traditionellt när ett operativsystem utvecklas funderar
utvecklarna på vad som krävs av ett bra operativsystem.
När Unix utvecklades var frågeställningen istället ''Vad
kan vi ta bort för att göra Unix bättre?'' Detta leder
till färre buggar då utvecklarna behöver skriva mindre
kod och det finns mindre plats för att göra fel.
Som följd av detta kan mer tid läggas där det behövs
och på så sätt hitta de existerande buggarna och utveckla
bättre (enklare) lösningar. Dessutom blir det enklare för
användaren att utföra arbeten, särskild sådana som
utvecklarna inte har tänkt på. Det blir dessutom enklare
för utvecklare.


\section*{Unix pipes}

Unix pipes är ett sätt för att få program att arbeta
tillsammans för att utföra ett arbete som beskrivs
av användaren. Unix pipes är det viktigaste i hela
Unix' design. När man använder Unix pipes, skriver
man en såkallad shellsekvens (programmet man skriv
i som startar programmen kallas för ett skal eller shell.)

Programmen som startas körs i parallellt med varandra
och kommunicerar med varandra i en kedja. Ett program
skriver ut och samtidigt så läser nästa program vad
programmet innan skriver ut. Varje program i sekvensen
utför en specifik uppgift och inget annat.
I början av kedjan kan antigen användaren (terminallen)
vara (användaren skriv till programmet) eller en fil som
läses in av det första programmet.
I slutat av kedjan kan antigen användaren (terminallen)
vara (användaren läser från programmet) eller en fil som
det sista programmet skriver till.

Exempel på program som brukar användas:
\newpage
\begin{itemize}
\item \textbf{sort}\\
Sorterar indata. Användaren kan välja ut datat
ska sorteras, till exempel alfabetisk ordning,
numerisk ordning och månadsordning.
\item \textbf{uniq} (unique)\\
Tar bort på varandra följande dubbletter.
Dubbletter som inte följer varandra tas inte bort,
därför vill man \emph{vanligtvis} sortera indatat
först. Man kan även specifier att antalet dubbletter
ska skrivas ut.
\item \textbf{sed} (stream editor)\\
Modifera indata efter ett eller flera angivna mönster.
\item \textbf{cat} (concatenate)\\
Skriv ut en eller flera filer.
\item \textbf{cut}\\
Dela ut varje rad i kolonner och välj ut vilka kolonner
som skall behållas.
\end{itemize}
Några exempel vi inte pratade om:
\begin{itemize}
\item \textbf{grep}\\
Liknanade \texttt{sed} men isället för att modifera rader
så behåller \texttt{grep}, eller tar bort, de rader som
beskrivs. Programmet kan även användas gör att behålla
delar av raderna istället för att behålla raderna fullständigt.
\item \textbf{diff} (difference)\\
Jämför filer för att hitta ändringar som har gjorts.
\item \textbf{comm} (compare)\\
Jämför två sorterade filer radvis.
Unix behövde inte ett rätt\-stav\-nings\-pro\-gram eftersom
\texttt{comm} kan jämföra varje ord i en fil mot en
ordlista:
\begin{verbatim}
grep -Po '[A-Za-z]+' < filen | dd conv=lcase | \
sort | uniq | comm -23 - /usr/share/dict/english
\end{verbatim}
\texttt{grep -Po '[A-Za-z]+'} skriver ut alla ord på varsin rad.\\
\texttt{dd conv=lcase} konverterar alla 'A'--'Z' till 'a'--'z'.
\item \textbf{shuf} (shuffle)\\
Skriv ut varje rad från indata i slumpmässig ordning.
\item \textbf{tac} (\texttt{cat} baklänges)\\
Skriv ut varje rad från indata i omvänd ordning.
\item \textbf{tee}\\
Skriv ut indata till nästa program och till andra filer.
\item \textbf{yes}\\
Skriv ut bokstaven 'y' om och om igen. Varje 'y' skrivs
ut på sin egna rad. Programmet slutar skriv när nästa
program avslutas, och pausas (automatiskt av operativsystemet)
om nästa program inte har lästa in tillräckligt många av
raderna som har skrivits. Vad som skall skrivas ut kan
specifieras av användaren.
\item \textbf{find}\\
Skriv ut alla filer på varsin rad, samt filer som
ligger i kataloger.
\item \textbf{xargs}\\
Kombineras ofta med \texttt{find}, varje rad i indata
används som ett argument till ett program som anges.
Exempelvis kan man utföra säker (irreversibel)
borttagning av alla filer i katalog genom att köra\\
\texttt{find katalogens\_namn | xargs shred -u}\\
(\texttt{shred -u a b c} utför säker borttagning
av filerna \texttt{a}, \texttt{b} och \texttt{c}.)
\end{itemize}
\*

\noindent
Under demonstrationen skriv vi en fil med namnet
''\texttt{m}''. Filen innehåll vilka månader vi
är födda:
\begin{verbatim}
feb
apr
apr
oct
nov
jul
feb
\end{verbatim}

\noindent
Med programmet \texttt{cat} kunde vi skriva ut
innehållet i filen: \texttt{cat m}
\\

\noindent
Programmen \texttt{cat} och \texttt{sort} kan
kombineras för att skriva ut en sorterad version
av filen: \texttt{cat m | sort}
\begin{verbatim}
apr
apr
feb
feb
jul
nov
oct
\end{verbatim}
\newpage

\noindent
Denna information kan även sorteras med
kännedomen att det är månader som är listade
i filen: \texttt{cat m | sort -M}
\begin{verbatim}
feb
feb
apr
apr
jul
oct
nov
\end{verbatim}

\noindent
Med hjälp av programmet \texttt{uniq} kan vi
ta bort dubbletter:\\\texttt{cat m | sort -M | uniq}
\begin{verbatim}
feb
apr
jul
oct
nov
\end{verbatim}

\noindent
Vi kan även säga till \texttt{uniq} att skriva
antalet dubbletter:\\\texttt{cat m | sort -M | uniq -c}
\begin{verbatim}
      2 feb
      2 apr
      1 jul
      1 oct
      1 nov
\end{verbatim}

\noindent
Istället för att använda programmet \texttt{cat}
för att skriva ut filen ''\texttt{m}'', kan vi
låta programmet \texttt{sort} läsa från filen
direkt: \texttt{sort -M < m | uniq -c}
\\

\noindent
Vi kan även skriv utdata till filen ''\texttt{mm}'' istället
för till terminalen:\\\texttt{sort -M < m | uniq -c > mm}
\\

\noindent
Med hjälp av ett litet program kan vi skriva ut ett
diagram av informationen:\\
\texttt{sort -M < m | uniq -c | ./hist}
\\\\
\tt
\reversevideo{fe}b (2)\\
\reversevideo{ap}r (2)\\
\reversevideo{j}ul (1)\\
\reversevideo{o}ct (1)\\
\reversevideo{n}ov (1)
\rm
\newpage

Programmet \texttt{./hist} är skrivet med samma språk
används i terminalen för att köra program. Detta gjorde
att programmet blev väldigt kort:

\begin{verbatim}
#!/bin/bash -e

while read -r line; do
 count=$(echo "${line}" | cut -d \  -f 1)
 title="$(echo "${line}" | cut -d \  -f 1 --complement) (${count})"
 if (( count > ${#title} )); then
  title="$(printf '%s%*.s' "${title}" $((count - ${#title})) '')"
 fi
 echo $'\e[07m'"${title::$count}"$'\e[27m'"${title:$count}"
done
\end{verbatim}

Det blev dessutom väldigt enkelt att förstå. Beroende
på hur mycket man använder terminallen skulle det ta
mellan ett halv år och två år att lära sig allt som krävs
för att skriva detta program, utan några kunskaper i
programmering. (Första rad i filen specifierar vilket
program som används för att köra koden)

Om man kör \texttt{sort -M | uniq -c | sort -n | tac | ./hist}
istället för \texttt{sort -M | uniq -c | ./hist} så sorteras
raderna i diagrammet på hur vanliga månaderna är istället
för på månadernas ordning. Det är alltså väldigt flexibellt
att program gör så lite som möjligt. Detta form av sortering
skulle inte vara möjligt om programmet var skrivet med
\texttt{sort | uniq -c} inbyggt.

\begin{verbatim}
#!/bin/bash -e

sed '/^$/d' | sort "$@" | uniq -c | while read -r line; do
 count=$(echo "${line}" | cut -d \  -f 1)
 title="$(echo "${line}" | cut -d \  -f 1 --complement) (${count})"
 if (( count > ${#title} )); then
  title="$(printf '%s%*.s' "${title}" $((count - ${#title})) '')"
 fi
 echo $'\e[07m'"${title::$count}"$'\e[27m'"${title:$count}"
done
\end{verbatim}

\noindent
(\texttt{sed '/\^\$/d'} tar bort tomma rader.)
\newpage

Att utföra samma arbeta som den inflexibla versionen
av diagramprogrammet utan att utnyttja Unix's design,
skulle kräva ett program med tre gånger så lång kod.

\begin{verbatim}
#!/bin/bash -e

declare -A events
while read -r line; do
 if [ "${line}" = '' ]; then
  continue
 fi
 if [ -z ${events["${line}"]} ]; then
  events["${line}"]=1
 else
  : $(( events["${line}"]++ ))
 fi
done
tmpfile="$(mktemp)"
for title in "${!events[@]}"; do
 echo "${title}";
done | sort "$@" > "${tmpfile}"
while read -r title; do
 count=${events["${title}"]}
 title="${title} (${count})"
 if (( count > ${#title} )); then
  title="$(printf '%s%*.s' "${title}" $((count - ${#title})) '')"
 fi
 echo $'\e[07m'"${title::$count}"$'\e[27m'"${title:$count}"
done < "${tmpfile}"
rm -- "${tmpfile}"
\end{verbatim}

Denna kod är inte bara mycket svårare att förstå,
den är också svårare att skriva och det är mycket
enkelt att köra slavfel (speciellt eftersom det
inte är skrivet i ett ordentligt programspråk.)
Denna kod använder dock programmet \texttt{sort}
för att sorter händelserna som ritas ut i diagrammet,
och låter användera välja hur de ska sorteras genom
programmet \texttt{sort}. Hade \texttt{sort} inte
utnyttjats med detta sätta hade programmet blivit
väldigt mycket längre.




\end{document}

