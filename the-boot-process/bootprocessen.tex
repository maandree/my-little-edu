\documentclass[12pt,a4paper]{article}
\usepackage{amsmath}
\usepackage{verbatim}
\usepackage{tabularx}
\usepackage[utf8]{inputenc}
\usepackage[pdftex]{graphicx}
%\usepackage[xetex]{graphicx}
%\usepackage{fontspec,xunicode}
\usepackage{color}

\newcommand{\reversevideo}[1]{\colorbox{black}{\!\phantom{X}\!\!\!\color{white}{#1\!}}}
\fontencoding{T1}
%\pagestyle{empty}
%\renewcommand{\familydefault}{\sfdefault}
%\renewcommand{\familydefault}{\rmdefault}


\begin{document}

\title{\Huge{\textbf{Bootprocessen}}}
\date{}
\author{}
\maketitle


\section*{BIOS}

(BIOS står för Basic Input/Output System och är det
första programmet som startar på datorn. Många moderkort
har sitt egna BIOS, men känt exempel som kan installeras
på de flesta datorerna är coreboot. Chromeböcker använder
coreboot.)
\\

\noindent
BIOS initiserar viktigt hårdvara såsom RAM-minnet.\footnote{Mer
specifict så initiseras RAM-minnet och chipset:et,
samt att hårdvara inbyggt i moderkortet letas upp.}
\\

\noindent
Sätter CPU:ns instruktionspekare till början av första hårddisken.
Användaren kan dock ofta välja vilken enhet som ska köras.
Innan detta sker kollar BIOS att verkligen finns ett program
i början av enheten som ska köras. Detta görs genom att kolla
att första instruktionen sätter värdet på CPU:ns första register
till 0.
\\


\section*{Boot manager}

Boot manager\footnote{Eller bootstrap manager då 'boot' är en
förkortning för 'bootstrap'.}, även kallat boot master record
code (MBR code) eller bootstrap code, är ett litet program
som ligger i början av en körbar enhet. Programmet är som mest
440 byte stort, eller 446 vilket inte rekommenderas.
\\

\noindent
Boot manager:n initiserar delar av CPU:en (stacken och segment
register.) Efter detta letar boot manager:n efter en körbar
\footnote{Kallas ofta bootpartition eller aktiv partition.}
primär partition\footnote{Visa väljer den första, några väljer
den sista, men de flesta ger ett felmeddelande om det finns mer
än en.} på hårddisken.
\\

\noindent
När boot manager:n är klar med alla andra av dess uppgifter
sätter den CPU:ns instruktionspekare till början av den körbara
partitionen.


\section*{Bootloader}

Bootloader, även kallat second-stage boot manager eller
på grund av förvirning\footnote{Det är även väldigt vanligt
att de som vet skillnanden säger fel.} boot manager, är ett
större program. Ofta mellan 1 och 2 MB.
\\

\noindent
Bootloader:n har oftast en egen partition (eller delar
partition med ett operativssytem), men vissa lägger sig
mellan partitionstabellen och den första partition.
På grund av detta brukar partitioneringsprogram inte
tillåta att en partition ligger innan den 2:a megabyten
på hårddisken.
\\

\noindent
Bootloader:ns uppgift är att tillåta användaren välja
vilket operativsystem som skall starts. Vissa
operativsystemskärnor kräver dock en bootloader
för att kunna startas.
\\

\noindent
En bootloader brukar även tillåta avändaren att
välja hur operativsystemet ska starts.
\\

\noindent
En bootloader brukar även installeras med en samling
små program som kan startas direkt utan ett
operativsystem. Exempel på sådana är minnestestsprograms,
hårdvaruinformations- och hårdvaruinspektionsprogram,
ett program som startar om datorn och ett program som
stänger av datorn.
\\

\noindent
En bootloader kan även start en annan bootloader,
detta kallas för kedjeladdning (chain loading), eller
en boot manager, till exempel en annan boot manager
som från en annan hårddisk. Kedjeladdning kan vara
nedvändigt för att starta operativsystem som inte
är skrivna för att kunna startas av vilken bootloader
som helt. Till exempel, om datorn har både GNU/Linux
och Windows installerat så startas först GNU/Linux's
bootloader. Om användaren väljer att starta Windows
så startar bootloader:n då Windows' bootloader efters
endast Windows' bootloader vet hur man startar Windows.


\section*{Operativsystemskärnan}

Operativsystemskärnan (operativ system kernel, eller
endast kernel) är operativsystemets program som har
hand om blanda annat att start andra program, låta
program kommunicera med varandra och låta flera
program köras samtidigt.
\\

\noindent
Operativsystemskärnan brukar lika som en fil på
bootloader:ns partition på hårddisken. (Detta är
bland annat så att man kan vilja vilken version
av kärnan som skall startas.)


\section*{initramfs/initrd/initcpio}

Nästa steg är frivilligt och brukar användas.
Detta steg heter i Unix-ter\-mi\-mo\-lo\-gi initramfs,
initrd eller ibland initcpio. initramfs är en
förkortning för 'initialisation RAM filesystem',
initrd är en förkortning för 'initialisation RAM
disc', och initcpio är en förkortning för
'initialisation \texttt{cpio}'. \texttt{cpio} är
en gammalt filarkiveringsprogram.
\\

\noindent
initramfs är ett väldigt minimal installation
av ett operativsystem. Istället för att vara
installerat på en partition på hårddisken är
den sparad som en \texttt{cpio}-fil, eller
liknande, undansparad på bootloader:ns partition.
Detta är så att man kan ha flera olika initramfs
installaderad som man kan välja mellan.
\\

\noindent
initramfs viktigtaste uppgifter är att dekryptera
partitioner och montera partitioner. Ett
Unix-liknande operativsystem kan man installeras
flera partition. initramfs är även viktigt om
man har mjukvaru-RAID eller någonting liknande
som flerdelar en partition över flera hårddiskar
utan hårdvarustöd.


\section*{init och rc}

\texttt{init} och rc är Unix-terminologi för det sista
steget. \texttt{init} är ett litet program sen ansvarar
(bland annat) för att starta rc. Det är idag okänt vad
rc står för, men det är ett generiskt namn på processen
som anvarar för att:
\begin{itemize}
\item \textbf{initiserar arbetsmiljön},\\
Detta innefattar bland annat information
om var man hittar alla program, och
montera partitioner.
\newpage
\item \textbf{starta daemoner} och\\
Detta är program som körs i bakgrunden.
Vanliga exempel du hittar på en dator
med ett Unix-liknande operativsystem är:
\begin{itemize}
\item \textbf{dhcpd} (Dynamic Host Configuration Protocol daemon)\\
Ger tillgång till nätverket.
\item \textbf{sshd} (Secure Shell daemon)\\
Möjliggör fjärrinloggning till datorn.
\item \textbf{udevd} (Userspace Device Management daemon)\\
Upptäcker när hårdvara kollas in eller tas bort.
\item \textbf{crond} (cron\footnote{Från 'chronos'.} daemon)\\
Kör program periodiskt, såsom att renas systemloggar.
\item \textbf{ntpd} (Network Time Protocol daemon)\\
Synkroniserar klockan på datorn över Internet mot atomklockor.
\item \textbf{cupsd} (Common Unix Printing System daemon)\\
Används för att använda eller dela med sig av skrivmaskiner.
\end{itemize}
\item \textbf{starta loginhanterare}
\end{itemize}


\end{document}

