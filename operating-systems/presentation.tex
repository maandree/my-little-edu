\documentclass[compress, final]{beamer}

\mode<presentation>
{
  \setbeamertemplate{background canvas}[vertical shading][bottom=white,top=structure.fg!25]
  
  \usetheme{Warsaw}
  
  \setbeamercovered{transparent}
  \setbeamertemplate{headline}{}
  %\setbeamertemplate{footline}{}
  %\setbeamersize{text margin left=0.5cm}
}

\usepackage[english]{babel}
\usepackage[utf8]{inputenc}
\usepackage[T1]{fontenc}
\usepackage{times}
\usepackage{verbatim}

\title{Skillnaden mellan operativsystem}
\date{}
\subject{Talks}
\beamerdefaultoverlayspecification{<+->}
\beamertemplatenavigationsymbolsempty
\begin{document}
\begin{frame}
  \titlepage
\end{frame}


\section{GNU}

\begin{frame}{GNU}
  \begin{minipage}[l]{0.60\textwidth}
    \textbf{GNU's not Unix} (GNU)
    \begin{itemize}
      \item Richard Stallman, 1983
      \item Frihet som mål
      \item Viss överförbarhet som ''mål''
      \item Unix var populärt och enkelt att ersätta
    \end{itemize}
  \end{minipage}
  \begin{minipage}[r]{0.35\textwidth}
    \pgfdeclareimage[width=\textwidth]{gnu}{gnu}
    \pgfuseimage{gnu}
  \end{minipage}
\end{frame}


\section{GNU Hurd}

\begin{frame}{GNU Hurd}
  \begin{minipage}[l]{0.60\textwidth}
    \textbf{GNU Hurd}
    \begin{itemize}
      \item Hird of Unix-Replacing Daemons
      \item Hurd of Interfaces Representing Depth
      \item Thomas Bushnell, 1990
      \item Kärnan till GNU
    \end{itemize}
  \end{minipage}
  \begin{minipage}[r]{0.35\textwidth}
    \pgfdeclareimage[width=\textwidth]{hurd}{hurd}
    \pgfuseimage{hurd}
  \end{minipage}
\end{frame}


\section{Linux}

\begin{frame}{Linux}
  \begin{minipage}[l]{0.60\textwidth}
    \textbf{Linux}
    \begin{itemize}
      \item Linus Torvalds, 1991
      \item Ville ha någonting hemma som liknar Unix
      \item Linux var ett arbetsnamn
      \item Endast kärna
    \end{itemize}
  \end{minipage}
  \begin{minipage}[r]{0.35\textwidth}
    \pgfdeclareimage[width=\textwidth]{linux}{linux}
    \pgfuseimage{linux}
  \end{minipage}
\end{frame}


\section{BSD}

\begin{frame}{BSD}
  \textbf{Berkeley Software Distribution} (BSD)
  \begin{itemize}
    \item Bill Joy, 1977
    \item Modifiera och vidareutveckla Unix för Berkeley
    \item Andra universitet var intresserade av BSD
  \end{itemize}
\end{frame}


\section{FreeBSD}

\begin{frame}{FreeBSD}
  \begin{minipage}[l]{0.60\textwidth}
    \textbf{FreeBSD}
    \begin{itemize}
      \item 1993
      \item Prestanda som mål
    \end{itemize}
  \end{minipage}
  \begin{minipage}[r]{0.35\textwidth}
    \pgfdeclareimage[width=\textwidth]{freebsd}{freebsd}
    \pgfuseimage{freebsd}
  \end{minipage}
\end{frame}


\section{NetBSD}

\begin{frame}{NetBSD}
  \begin{minipage}[l]{0.60\textwidth}
    \textbf{NetBSD}
    \begin{itemize}
      \item Chris Demetriou, Theo de Raadt, Adam Glass och Charles Hannum, 1993
      \item Överförbarhet som mål
    \end{itemize}
  \end{minipage}
  \begin{minipage}[r]{0.35\textwidth}
    \pgfdeclareimage[width=\textwidth]{netbsd}{netbsd}
    \pgfuseimage{netbsd}
  \end{minipage}
\end{frame}


\section{OpenBSD}

\begin{frame}{OpenBSD}
  \begin{minipage}[l]{0.60\textwidth}
    \textbf{OpenBSD}
    \begin{itemize}
      \item Theo de Raadt, 1995
      \item Säkerhet som mål
    \end{itemize}
  \end{minipage}
  \begin{minipage}[r]{0.35\textwidth}
    \pgfdeclareimage[width=\textwidth]{openbsd}{openbsd}
    \pgfuseimage{openbsd}
  \end{minipage}
\end{frame}


\section{DragonFly BSD}

\begin{frame}{DragonFly BSD}
  \begin{minipage}[l]{0.60\textwidth}
    \textbf{DragonFly BSD}
    \begin{itemize}
      \item Matt Dillon, 2003
      \item Skalbarhet som mål
    \end{itemize}
  \end{minipage}
  \begin{minipage}[r]{0.35\textwidth}
    \pgfdeclareimage[width=\textwidth]{dragonflybsd}{dragonflybsd}
    \pgfuseimage{dragonflybsd}
  \end{minipage}
\end{frame}


\section{MINIX}

\begin{frame}{MINIX}
  \begin{minipage}[l]{0.60\textwidth}
    \textbf{MINIX}
    \begin{itemize}
      \item Andrew Tanenbaum, 1987
      \item Utvecklat för utlärning om operativsystem
      \item Stabilitet som nytt mål
    \end{itemize}
  \end{minipage}
  \begin{minipage}[r]{0.35\textwidth}
    \pgfdeclareimage[width=\textwidth]{minix}{minix}
    \pgfuseimage{minix}
  \end{minipage}
\end{frame}


\section{Plan 9 from Bell Labs}

\begin{frame}{Plan 9 from Bell Labs}
  \begin{minipage}[l]{0.60\textwidth}
    \textbf{Plan 9 from Bell Labs}
    \begin{itemize}
      \item Bell Labs, 1980-talet
      \item Designat för distribuerad databehandling
    \end{itemize}
  \end{minipage}
  \begin{minipage}[r]{0.35\textwidth}
    \pgfdeclareimage[width=\textwidth]{plan9}{plan9}
    \pgfuseimage{plan9}
  \end{minipage}
\end{frame}


\section{POSIX}

\begin{frame}{POSIX}
  \textbf{Portable Operating System Interface} (POSIX)
  \begin{itemize}
    \item Institute of Electrical and Electronics Engineer, 1988
    \item Standardisering av Unix-liknande operativsystem
  \end{itemize}
\end{frame}


\section{Megalitisk kärna}

\begin{frame}{Megalitisk kärna}
  \textbf{Megalitisk kärna}
  \begin{itemize}
    \item Alla program är inbyggda i kärnan
    \item Vanliga för realtidsoperativsystem
  \end{itemize}
\end{frame}


\section{Monolitisk kärna}

\begin{frame}{Monolitisk kärna}
  \textbf{Monolitisk kärna}
  \begin{itemize}
    \item Traditionell design
    \item Har hand om bland annat
    \begin{itemize}
      \item Filsystem
      \item Systemanrop
      \item Interprocesskommunikation
      \item Processplanering
      \item Virtuellt minne
      \item Hårdvara
      \item Drivrutiner
    \end{itemize}
    \item Är vanligtvis modulära
  \end{itemize}
\end{frame}


\section{Monolitisk kärna}

\begin{frame}{Hybridkärna}
  \textbf{Hybridkärna}
  \begin{itemize}
    \item Monolitisk?
    \item Har hand om bland annat
    \begin{itemize}
      \item Interprocesskommunikation
      \item Processplanering
      \item Virtuellt minne
      \item Hårdvara
    \end{itemize}
    \item Exempel på implementationer
    \begin{itemize}
      \item Windows NT
      \item XNU (OS X)
      \item Darwin (OS X)
      \item Plan 9 from Bell Labs
      \item Haiku
    \end{itemize}
  \end{itemize}
\end{frame}


\section{Mikrokärna}

\begin{frame}{Mikrokärna}
  \textbf{Mikrokärna}
  \begin{itemize}
    \item Har hand om bland annat
    \begin{itemize}
      \item Minimal Interprocesskommunikation
      \item Processplanering
      \item Virtuellt minne
    \end{itemize}
    \item Andra viktiga funktioner hanteras av sevrar
    \item Exempel på implementationer
    \begin{itemize}
      \item Mach
      \item GNU Hurd
      \item L4
      \item MINIX
      \item AmigaOS
    \end{itemize}
  \end{itemize}
\end{frame}


\section{Nano-/pikokärna}

\begin{frame}{Nano-/pikokärna}
  \textbf{Nano-/pikokärna}
  \begin{itemize}
    \item Mikrokärna?
    \item Exempel på implementationer
    \begin{itemize}
      \item Mac OS 9
    \end{itemize}
  \end{itemize}
\end{frame}


\section{Exokärna}

\begin{frame}{Exokärna}
  \textbf{Exokärna}
  \begin{itemize}
    \item Gör så gott som ingenting
    \item Mål: biblioteksoperativssystem
    \item Forskningsområde
  \end{itemize}
\end{frame}


\section{Virtualiserande kärna}

\begin{frame}{Virtualiserande kärna}
  \textbf{Virtualiserande kärna}, typ 1-hypervisor
  \begin{itemize}
    \item Exokärna
    \item Tillåter flera operativsystem att köras samtidigt
    \item Exempel på implementationer
    \begin{itemize}
      \item Hyper-V
    \end{itemize}
  \end{itemize}
\end{frame}


\section{Cachekärna}

\begin{frame}{Cachekärna}
  \textbf{Cachekärna}
  \begin{itemize}
    \item Exokärna
    \item Exempel på implementationer
    \begin{itemize}
      \item Stanford cache kernel
    \end{itemize}
  \end{itemize}
\end{frame}


\end{document}

